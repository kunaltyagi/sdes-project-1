\documentclass[12pt, a4paper]{article}

% Author: Kunal Tyagi

\usepackage{amsfonts,amssymb,amsthm, amsmath}
\usepackage[square,numbers]{natbib}
\usepackage{graphicx}
\usepackage{wrapfig}
\usepackage{caption}
\usepackage{hyperref}
% \captionsetup[figure]{labelformat=empty}

%opening
\title{Van-der-Pol Oscillator}
\author{Kunal Tyagi}
\date{\today}

\graphicspath{{figure/}}
\renewcommand\bibname{References}

\begin{document}
\maketitle
The Van der Pol oscillator is a non-conservative oscillator with
non-linear damping.  It evolves in time according to the second-order
differential equation \cite{wiki, slides}:
\begin{equation}
    \frac{d^2x}{dt} - \mu (1 - x^2)\frac{dx}{dt} + x = 0
\end{equation}
where $x$ is the position coordinate—which is a function of the time $t$, and
$\mu$ is a scalar parameter indicating the nonlinearity and the strength of
the damping.

These plots were generated by using the transformation $y = \dot(x)$
giving us the following set of Ordinary Differential Equations, which
was solved using \textbf{scipy}.
\begin{align}
    \dot(x) &= y\\
    \dot{y} &= \mu(1- x^2)y - x
\end{align}

The following plots were generated by changing the initial position as
well as the damping factor. Here, time is in \textbf{centi-seconds}.

% This is a python generated file. Do not edit since it will be rewritten during make
\begin{figure}[h]
    \centering
    \begin{minipage}{0.45\textwidth}
        \centering
        \includegraphics[width=\textwidth]{1}
        \caption{Figure 1}
        \label{fig:fig1}
    \end{minipage}\hfill
    \begin{minipage}{0.45\textwidth}
        \centering
        \includegraphics[width=\textwidth]{2}
        \caption{Figure 2}
        \label{fig:fig2}
    \end{minipage}
\end{figure}

\begin{figure}[h]
    \centering
    \begin{minipage}{0.45\textwidth}
        \centering
        \includegraphics[width=\textwidth]{3}
        \caption{Figure 3}
        \label{fig:fig3}
    \end{minipage}\hfill
    \begin{minipage}{0.45\textwidth}
        \centering
        \includegraphics[width=\textwidth]{4}
        \caption{Figure 4}
        \label{fig:fig4}
    \end{minipage}
\end{figure}

\begin{figure}[h]
    \centering
    \begin{minipage}{0.45\textwidth}
        \centering
        \includegraphics[width=\textwidth]{5}
        \caption{Figure 5}
        \label{fig:fig5}
    \end{minipage}\hfill
    \begin{minipage}{0.45\textwidth}
        \centering
        \includegraphics[width=\textwidth]{6}
        \caption{Figure 6}
        \label{fig:fig6}
    \end{minipage}
\end{figure}


From the plots it is clear that there are two regimes in the oscillator:
\begin{itemize}
        \item At $\mu$ = 0, it is the equation of a simple harmonic
            oscillator
        \item At $\mu > $0, near $\frac{dx}{dt} = 0$, it is unstable, and
            stable otherwise. See Figure \ref{fig:variation}
\end{itemize}

The code for this can be found at
\href{https://github.com/kunaltyagi/sdes-project-1}{this repository on
github}.
\begin{figure}[ht]
    \centering
    \includegraphics[width=0.5\textwidth]{variation}
    \caption{Variation on Oscillator with change in $\mu$}
    \label{fig:variation}
\end{figure}
\bibliographystyle{abbrv}
\bibliography{latex/reference}{}
\end{document}
